\chapter{编写泛型代码的改进}\label{ch33}
C++17引入了很多辅助工具来帮助实现泛型代码和库。

注意我们已经在在\nameref{ch21}这一章中介绍了一些新的类型特征。


\section{\texttt{std::invoke<>()}}\label{ch33.1}
新的工具\texttt{std::invoke<>()}是一个新的辅助函数,
它被用于编写调用一个可调用对象的代码,可调用对象包括函数、lambda、
有\texttt{operator()}的函数对象、成员函数。

这里有一个辅助函数的例子演示怎么使用它:
\inputcodefile{tmpl/invoke.hpp}
你传递给\texttt{call()}的第一个参数,将会按照如下方式使用剩余的参数进行调用:
\begin{itemize}
    \item 如果可调用对象是一个成员函数的指针,将使用剩余参数中的第一个参数作为对象调用成员函数,
    所有其他参数被用作调用的参数。
    \item 否则,可调用对象会把剩余参数用作自己的参数进行调用。
\end{itemize}
例如:
\inputcodefile{tmpl/invoke.cpp}
注意在不指明要调用哪个版本的情况下调用重载函数将导致错误:
\begin{lstlisting}
    call(&decltype(vals)::resize, vals, 5);     // ERROR:resize()被重载了

    call<void(decltype(vals)::*)(std::size_t)>(&decltype(vals)::resize, vals, 5);   // OK
\end{lstlisting}
还要注意调用函数模板需要显式实例化。如果\texttt{print()}是一个模板:
\begin{lstlisting}
    template<typename T>
    void print(const T& coll)
    {
        std::cout << "elems: ";
        for (const auto& elem : coll) {
            std::cout << elem << ' ';
        }
        std::cout << '\n';
    }
\end{lstlisting}
那么当你将它传给\texttt{call}时必须显式指明模板参数:
\begin{lstlisting}
    call(print, vals);                      // ERROR:不能推导出模板参数T

    call(print<std::vector<int>>, vals);    // OK
\end{lstlisting}
最后,注意根据移动语义的规则,转发一个调用的结果需要使用\texttt{decltype(auto)}来
\emph{完美返回}返回值到调用者:
\begin{lstlisting}
    template<typename Callable, typename... Args>
    decltype(auto) call(Callable&& op, Args&&.. args)
    {
        return std::invoke(std::forward<Callable>(op),      // 调用传入的可调用对象
                           std::forward<Args>(args)...);    // 以传入的其他参数为参数
    }
\end{lstlisting}


\section{\texttt{std::bool\_constant<>}}\label{ch33.2}
如果一个特征返回bool值,那么它们现在使用了新的模板别名\texttt{bool\_constant<>}:
\begin{lstlisting}
    namespace std {
        template<bool B>
        using bool_constant = integral_constant<bool, B>;   // 自从C++17起
        using true_type = bool_constant<true>;
        using false_type = bool_constant<false>;
    }
\end{lstlisting}
在C++17之前,你必须直接使用\texttt{integral\_constant<>},这意味着\texttt{true\_type}和
\texttt{false\_type}按照如下方式定义:
\begin{lstlisting}
    namespace std {
        using true_type = integral_constant<bool, true>;
        using false_type = integral_constant<bool, false>;
    }
\end{lstlisting}
bool特征仍然是在满足特定属性时继承\texttt{std::true\_type},
在不满足时继承\texttt{std::false\_type}。例如:
\begin{lstlisting}
    // 主模板:T不是void类型时
    template<typename T>
    struct IsVoid : std::false_type {
    };

    // 为类型void的特化
    template<>
    struct IsVoid<void> : std::true_type {
    };
\end{lstlisting}
然而,你现在可以通过派生自\texttt{bool\_constant<>}来定义自己的类型特征,
只需要制定相应的编译期表达式作为一个bool条件。例如:
\begin{lstlisting}
    template<typename T>
    struct IsLargerThanInt : std::bool_constant<(sizeof(T) > sizeof(int))> {
    }
\end{lstlisting}
之后你可以使用这样一个特征来在编译器判断一个类型是否大于\texttt{int}:
\begin{lstlisting}
    template<typename T>
    void foo(T x)
    {
        if constexpr(IsLargerThanInt<T>::value) {
            ...
        }
    }
\end{lstlisting}
通过添加相应的\hyperref[ch21.1]{带有后缀\_v的变量模板}作为\nameref{ch3}:
\begin{lstlisting}
    template<typename T>
    inline static constexpr auto IsLargerThanInt_v = IsLargerThanInt<T>::value;
\end{lstlisting}
你可以把这个特征的使用缩短为如下形式:
\begin{lstlisting}
    template<typename T>
    void foo(T x)
    {
        if constexpr(IsLargerThanInt_v<T>) {
            ...
        }
    }
\end{lstlisting}
作为另一个例子,我们可以定义一个如下的特征来粗略的检查
一个类型\texttt{T}的移动构造函数是否保证不抛出异常:
\begin{lstlisting}
    template<typename T>
    struct IsNothrowMoveConsructibleT : std::bool_constant<noexcept(T(std::declval<T>()))> {
    };
\end{lstlisting}


\section{\texttt{std::void\_t<>}}\label{ch33.3}
还有一个很小但很有用的辅助定义类型特征的工具在C++17中被标准化了:\texttt{std::void\_t<>}。
它简单的按照如下形式定义:
\begin{lstlisting}
    namespace std {
        template<typename...> using void_t = void;
    }
\end{lstlisting}
也就是说,对于任何可变模板参数列表它都会返回\texttt{void}。
如果我们只想在参数列表中处理类型时这会很有用。

它的主要应用就是当定义新的类型特征时检查条件。
下面的例子演示了它的应用:
\begin{lstlisting}
    #include <utility>      // for declval<>
    #include <type_traits>  // for true_type, false_type, void_t

    // 主模板:
    template<typename, typename = std::void_t<>>
    struct HasVarious : std::false_type {
    };

    // 部分特化(may be SFINAE'd away):
    template<typename T>
    struct HasVarious<T, std::void_t<decltype(std::declval<T>().begin()),
                                     typename T::difference_type,
                                     typename T::iterator>>
        : trye_type {
    };
\end{lstlisting}
这里,我们定义了一个新的类型特征\texttt{HasVariousT<>},它会检查如下三个条件:
\begin{itemize}
    \item 该类型有成员函数\texttt{begin()}吗?
    \item 该类型有类型成员\texttt{difference\_type}吗?
    \item 该类型有类型成员\texttt{iterator}吗?
\end{itemize}
只有当对于类型\texttt{T}所有相应的条件都有效时才会使用部分特化版本。
在这种情况下,它的特化程度比主模板更高所以会使用它,
并且因为我们从\hyperref[ch33.2]{\texttt{std::true\_type}},
检查这个特征的值将返回\texttt{true}:
\begin{lstlisting}
    if constexpr (HasVarious<T>::value) {
        ...
    }
\end{lstlisting}
如果任何表达式导致无效代码(即\texttt{T}没有\texttt{begin()}、或者
没有类型成员\texttt{difference\_type}、或者没有类型成员\texttt{iterator}),
部分特化版会\emph{SFINAE'd away},这意味着根据
\emph{代换失败不是错误(substitution failure is not an error)}规则它会被忽略。
之后,只有主模板可以使用,它派生自\texttt{std::false\_type},如果检查它的值
会返回\texttt{false}。

使用这种方式,你可以使用\texttt{std::void\_t}来轻易的定义其他检查一个或多个条件的特征,
这些条件包括是否存在某个成员或操作或者某个成员或操作的能力。
参见\hyperref[ch30.1.2.2]{\texttt{HasDelete<>}}获取另一个例子。


\section{后记}
\texttt{std::invoke<>}由Tomasz Kaminski在\url{https://wg21.link/n3727}中首次提出。
最终被接受的是Tomasz Kaminski发表于\url{https://wg21.link/4169}的提案。
之后根据Daniel Krügler、Pablo Halpern、Jonathan Wakely在
\url{https://wg21.link/p0604r0}中的提议使用\nameref{ch21.2.0.5}修改了它的返回值类型。

\texttt{std::bool\_constant<>}由Zhihao Yuan在\url{https://wg21.link/n4334}中
首次提出。最终被接受的是Zhihao Yuan发表于\url{https://wg21.link/n4389}的提案。

\texttt{std::void\_t<>}因Walter E. Brown在\url{https://wg21.link/n3911}中
的提议而被采纳。