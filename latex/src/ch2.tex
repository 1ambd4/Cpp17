\section{带初始化的\texttt{if}和\texttt{switch}语句}\label{ch2}
\texttt{if}和\texttt{switch}语句现在允许在条件表达式里添加一条初始化语句。
例如,你可以写出如下代码:
\begin{lstlisting}
    if (status s = check(); s != status::success) {
        return s;
    }
\end{lstlisting}
其中的初始化语句是:
\begin{lstlisting}
    status s = check();
\end{lstlisting}
它初始化了\texttt{s},\texttt{s}将在整个\texttt{if}语句中有效(包括\texttt{else}分支里)。

\subsection{带初始化的\texttt{if}语句}
在\texttt{if}语句的条件表达式里定义的变量将在整个\texttt{if}语句中有效,
包括\emph{then}和\emph{else}部分。例如:
\begin{lstlisting}
    if (std::ofstream strm = getLogStrm(); coll.empty()) {
        strm << "<no data>\n";
    }
    else {
        for (const auto& elem : coll) {
            strm << elem << '\n';
        }
    }
    // strm不再有效
\end{lstlisting}
在整个\texttt{if}语句结束时\texttt{strm}的析构函数会被调用。
另一个例子是关于锁的使用,假设我们要在并发的环境中执行一个依赖某些条件的任务:
\begin{lstlisting}
    if (std::lock_guard<std::mutex> lg{collMutex}; !coll.empty()) {
        std::cout << coll.front() << '\n';
    }
\end{lstlisting}
这个例子中,如果使用\nameref{ch9},可以改写成如下代码:
\begin{lstlisting}
    if (std::lock_guard lg{collMutex}; !coll.empty()) {
        std::cout << coll.front() << '\n';
    }
\end{lstlisting}
上面的代码等价于:
\begin{lstlisting}
    {
        std::lock_guard<std::mutex> lg{collMutex};
        if (!coll.empty()) {
            std::cout << coll.front() << '\n';
        }
    }
\end{lstlisting}
细微的区别在于前者中\texttt{lg}在\texttt{if}语句的作用域之内定义,
和条件语句在相同的作用域。

注意这个特性的效果和传统\texttt{for}循环里的初始化语句完全相同。
上面的例子中为了让\texttt{lock\_guard}生效,必须在初始化语句里明确声明一个变量名,
否则它就是一个临时变量,会在创建之后就立即销毁。因此,初始化一个没有变量名的临时
\texttt{lock\_guard}是一个逻辑错误,因为当执行到条件语句时锁就已经被释放了:
\begin{lstlisting}
    if (std::lock_guard<std::mutex>{collMutex};     // 运行时ERROR
        !coll.empty()) {                    // 锁已经被释放了
        std::cout << coll.front() << '\n';  // 锁已经被释放了
    }
\end{lstlisting}
原则上讲,使用简单的\texttt{\_}作为变量名就已经足够了:
\begin{lstlisting}
    if (std::lock_guard<std::mutex> _{collMutex};   // OK,但是...
        !coll.empty()) {
        std::cout << coll.front() << '\n';
    }
\end{lstlisting}
你也可以同时声明多个变量,并且可以在声明时初始化:
\begin{lstlisting}
    if (auto x = qqq1(), y = qqq2(); x != y) {
        std::cout << "return values " << x << " and " << y << "differ\n";
    }
\end{lstlisting}
或者:
\begin{lstlisting}
    if (auto x{qqq1()}, y{qqq2()}; x != y) {
        std::cout << "return values " << x << " and " << y << "differ\n";
    }
\end{lstlisting}
另一个例子则是向\texttt{map}或者\texttt{unordered map}插入元素。
你可以像下面这样检查是否成功:
\begin{lstlisting}
    std::map<std::string, int> coll;
    ...
    if (auto [pos, ok] = coll.insert({"new", 42}); !ok) {
        // 如果插入失败,用pos处理错误
        const auto& [key, val] = *pos;
        std::cout << "already there: " << key << '\n';
    }
\end{lstlisting}
这里,我们用了\nameref{ch1}来给返回的\texttt{pos}指向的值声明了新的名称,
而不是使用\texttt{first}和\texttt{second}成员。在C++17之前,相应的处理代码必须像下面这样写:
\begin{lstlisting}
    auto ret = coll.insert({"new", 42});
    if (!ret.second) {
        // 如果插入失败,用ret.first处理错误
        const auto& elem = *(ret.first);
        std::cout << "already there: " << elem.first << '\n';
    }
\end{lstlisting}
注意这个拓展也适用于\nameref{ch10}特性。

\subsection{带初始化的\texttt{switch}语句}
通过使用带初始化的\texttt{switch}语句,我们可以在控制流之前初始化一个对象/实体。
例如,我们可以先声明一个\nameref{ch20.2.3},然后再根据它的类别进行处理:
\begin{lstlisting}
    namespace fs = std::filesystem;
    ...
    switch (fs::path p{name}; status(p).type()) {
        case fs::file_type::not_found:
            std::cout << p << " not found\n";
            break;
        case fs::file_type::directory:
            std::cout << p << ":\n";
            for (const auto& e : std::filesystem::directory_iterator{p}) {
                std::cout << "- " << e.path() << '\n';
            }
            break;
        default:
            std::cout << p << " exists\n";
            break;
    }
\end{lstlisting}
这里,初始化的路径\texttt{p}可以在整个\texttt{switch}语句中使用。

\subsection{后记}
带初始化语句的\texttt{if}和\texttt{switch}语句最早由Thomas Köppe
在\url{https://wg21.link/p0305r0}中提出,一开始只是提到了扩展\texttt{if}语句。
最终被接受的正式提案由Thomas Köpped发表于\url{https://wg21.link/p0305r1}。