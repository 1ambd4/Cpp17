\chapter{类型特征扩展}\label{ch21}
C++17扩展了类型特征(标准类型函数)的通用能力并引入了一些新的类型特征。


\section{类型特征后缀\texttt{\_v}}\label{ch21.1}
自从C++17起,你可以对所有返回值的类型特征使用后缀\texttt{\_v}
(就像你可以为所有返回类型的类型特征使用\texttt{\_t}一样)。
例如,对于任何类型\texttt{T},你现在可以写:
\begin{lstlisting}
    std::is_const_v<T>      // 自从C++17起
\end{lstlisting}
来代替:
\begin{lstlisting}
    std::is_const<T>::value // 自从C++11起
\end{lstlisting}
这适用于所有返回值的类型特征。方法是为每一个标准类型特征定义一个相应的新的变量模板。例如:
\begin{lstlisting}
    namespace std {
        template<typename T>
        constexpr bool is_const_v = is_const<T>::value;
    }
\end{lstlisting}
这样一个类型特征可以也用做运行时的条件表达式:
\begin{lstlisting}
    if (std::is_signed_v<char>) {
        ...
    }
\end{lstlisting}
然而,因为这些类型特征是在编译期求值,所以你也可以在编译期使用它们
(例如,在一个\nameref{ch10}中):
\begin{lstlisting}
    if constexpr (std::is_signed_v<char>) {
        ...
    }
\end{lstlisting}
另一个应用是用来支持不同的模板实例化:
\begin{lstlisting}
    // 类C<T>的主模板
    template<typename T, bool = std::is_pointer_v<T>>
    class C {
        ...
    };

    // 指针类型的偏特化版本:
    template<typename T>
    class C<T, true> {
        ...
    };
\end{lstlisting}
这里,类\texttt{C}为指针类型提供了一个偏特化版本。

后缀\texttt{\_v}也可以用于返回非bool类型的类型特征,例如\texttt{std::extent<>},
返回原生数组的某一个维度的大小:
\begin{lstlisting}
    int a[5][7];
    std::cout << std::extent_v<decltype(a)> << '\n';    // 打印出5
    std::cout << std::extent_v<decltype(a), 1> << '\n'; // 打印出7
\end{lstlisting}


\section{新的类型特征}
表\hyperref[t21.1]{新的类型特征}列出了C++17引入的新类型特征。
\begin{table}[htb]
    \centering
    \begin{tabular}{l|p{0.45\textwidth}}
        \hline
        \textbf{特征}                                        & \textbf{效果}                                   \\
        \hline
        \texttt{is\_aggregate<T>}                          & 是否是聚合体类型                                      \\
        \texttt{is\_swappable<T>}                          & 该类型是否能调用\texttt{swap()}                       \\
        \texttt{is\_nothrow\_swappable<T>}                 & 该类型是否能调用\texttt{swap()}并且该操作不会抛出异常            \\
        \texttt{is\_swappable\_with<T1, T2>}               & 特定值类型的这两种类型是否能调用\texttt{swap()}               \\
        \texttt{is\_nothrow\_swappable\_with<T1, T2>}      & 特定值类型的这两种类型是否能调用\texttt{swap()}并且该操作不会抛出异常    \\
        \texttt{has\_unique\_object\_representations<T>}   & 是否该类型的两个值相等的对象在内存中的表示也一样                      \\
        \texttt{is\_invocable<T, Args...>}                 & 该类型是否可以用\emph{Args...}调用                      \\
        \texttt{is\_nothrow\_invocable<T, Args...>}        & 该类型是否可以用\emph{Args...}调用,并且该操作不会抛出异常          \\
        \texttt{is\_invocable\_r<RT, T, Args...>}          & 该类型是否可以用\emph{Args...}调用并返回\emph{RT}类型        \\
        \texttt{is\_nothrow\_invocable\_r<RT, T, Args...>} & 该类型是否可以用\emph{Args...}调用并返回\emph{RT}类型且不会抛出异常 \\
        \texttt{invoke\_result<T, Args...>}                & 用\emph{Args...}作为实参进行调用会返回的类型                 \\
        \texttt{conjunction<B...>}                         & 对bool特征\emph{B...}进行逻辑与运算                     \\
        \texttt{disjunction<B...>}                         & 对bool特征\emph{B...}进行逻辑或运算                     \\
        \texttt{negation<B>}                               & 对bool特征B进行非运算                                 \\
        \texttt{is\_execution\_policy<T>}                  & 是否是\nameref{ch22.2}类型                         \\
        \hline
    \end{tabular}
    \caption{新的类型特征}
    \label{t21.1}
\end{table}

另外,\texttt{is\_literal\_type<>}和\texttt{result\_of<>}自从C++17起被废弃。
下面的段落将详细讨论这些特征。

\subsubsection{类型特征\texttt{is\_aggregate<>}}\label{ch21.2.0.1}
\begin{itemize}
    \item[] \texttt{std::\textbf{is\_aggregate<T>}::value}
\end{itemize}
返回\emph{T}是否是\hyperref[ch4.3]{聚合体类型}:
\begin{lstlisting}
    template<typename T>
    struct D : std::string, std::complex<T> {
        std::string data;
    };

    D<float> s{{"hello"}, {4.5, 6.7}, "world"};         // 自从C++17起OK
    std::cout << std::is_aggregate<decltype(s)>::value; // 输出:1(true)
\end{lstlisting}

\subsubsection{类型特征\texttt{is\_swappable<>}和\texttt{is\_swappable\_with<>}}
\begin{itemize}
    \item[] \texttt{std::\textbf{is\_swappable<T>}::value}
    \item[] \texttt{std::\textbf{is\_nothrow\_swappable<T>}::value}
    \item[] \texttt{std::\textbf{is\_swappable\_with<T1, T2>}::value}
    \item[] \texttt{std::\textbf{is\_nothrow\_swappable\_with<T1, T2>}::value}
\end{itemize}
在以下情况下返回true:
\begin{itemize}
    \item 类型\texttt{T}的左值可以被交换,或者
    \item 类型\texttt{T1}和\texttt{T2}的值类型可以交换
\end{itemize}
(使用\texttt{nothrow}形式时还要保证不会抛出异常)。

注意\texttt{is\_swappable<>}或\texttt{is\_nothrow\_swappable<>}检查你是否
可以交换某个指定类型的值(检查该类型的左值)。
相反,\texttt{is\_swappable\_with<>}和\texttt{is\_nothrow\_swappable\_with<>}还
会考虑值类型。也就是说:
\begin{lstlisting}
    is_swappable_v<int>             // 返回true
\end{lstlisting}
等价于
\begin{lstlisting}
    is_swappable_with_v<int&, int&> // 返回true(和上边等价)
\end{lstlisting}
而:
\begin{lstlisting}
    is_swappable_with_v<int, int>   // 返回false
\end{lstlisting}
将会返回false,因为你不能调用\texttt{std::swap(42, 77)}。

例如:
\begin{lstlisting}
    is_swappable_v<std::string>     // 返回true
    is_swappable_v<std::string&>    // 返回true
    is_swappable_v<std::string&&>   // 返回true
    is_swappable_v<void>            // 返回false
    is_swappable_v<void*>           // 返回true
    is_swappable_v<char[]>          // 返回false

    is_swappable_with_v<std::string, std::string>       // 返回false
    is_swappable_with_v<std::string&, std::string&>     // 返回true
    is_swappable_with_v<std::string&&, std::string&&>   // 返回false
\end{lstlisting}

\subsubsection{类型特征\texttt{has\_unique\_object\_representations<>}}
\begin{itemize}
    \item[] \texttt{std::\textbf{has\_unique\_object\_representations<T>}::value}
\end{itemize}
当任意两个值相同的类型\texttt{T}的对象在内存中的表示都相同时返回true。
也就是说,两个相同的值在内存中总是有相同的字节序列。

有这种属性的对象可以通过对字节序列哈希来得到对象的哈希值
(不用担心某些不参与比较的位可能不同的情况)。

\subsubsection{类型特征\texttt{is\_invocable<>}和\texttt{is\_invocable\_r<>}}
\begin{itemize}
    \item[] \texttt{std::\textbf{is\_invocable<T, Args...>}::value}
    \item[] \texttt{std::\textbf{is\_nothrow\_invocable<T, Args...>}::value}
    \item[] \texttt{std::\textbf{is\_invocable\_r<Ret, T, Args...>}::value}
    \item[] \texttt{std::\textbf{is\_nothrow\_invocable\_r<Ret, T, Args...>}::value}
\end{itemize}
当你能以\texttt{Args...}为实参调用\texttt{T}类型的对象并且返回值可以转换为\texttt{Ret}
类型(并且保证不抛出异常)时返回true。
也就是说,我们可以使用这些特征来测试我们是否可以用\texttt{Args...}为实参调用
(直接调用或者通过\hyperref[ch33.1]{\texttt{std::invoke()}})
\texttt{T}类型的可调用对象并返回\texttt{Ret}类型的值。

例如,如下定义:
\begin{lstlisting}
    struct C {
        bool operator() (int) const {
            return true;
        }
    };
\end{lstlisting}
会导致下列结果:
\begin{lstlisting}
    std::is_invocable<C>::value                             // false
    std::is_invocable<C, int>::value                        // true
    std::is_invocable<int*>::value                          // false
    std::is_invocable<int(*)()>::value                      // true

    std::is_invocable_r<bool, C, int>::value                // true
    std::is_invocable_r<int, C, long>::value                // true
    std::is_invocable_r<void, C, int>::value                // true
    std::is_invocable_r<char*, C, int>::value               // false
    std::is_invocable_r<long, int(*)(int)>::value           // false
    std::is_invocable_r<long, int(*)(int), int>::value      // true
    std::is_invocable_r<long, int(*)(int), double>::value   // true
\end{lstlisting}

\subsubsection{类型特征\texttt{invoke\_result<>}}\label{ch21.2.0.5}
\begin{itemize}
    \item[] \texttt{std::\textbf{invoke\_result<T, Args...>}::type}
\end{itemize}
返回当使用实参\texttt{Args...}调用\texttt{T}类型的对象时会返回的类型。
也就是说,我们可以使用这个特征来获知如果使用\texttt{Args...}调用\texttt{T}类型的对象时
将会返回的类型。

这个类型特征替换了\texttt{result\_of<>},后者现在不应该再使用。

例如:
\begin{lstlisting}
    std::string foo(int);

    using T1 = std::invoke_result_t<decltype(foo), int>;    // T1是std::string

    struct ABC {
        virtual ~ABC() = 0;
        void operator() (int) const {
        }
    };

    using T2 = typename std::invoke_result<ABC, int>::type; // T2是void
\end{lstlisting}

\subsubsection{bool类型特征的逻辑操作}
表\hyperref[t21.2]{组合其他类型特征的类型特征}列出了对bool类型类征(几乎所有的返回bool类型值的标准类型特征)
进行逻辑组合的类型特征。
\begin{table}[htb]
    \centering
    \begin{tabular}{l|l}
        \hline
        \textbf{特征}                & \textbf{效果}                      \\
        \hline
        \texttt{conjunction<B...>} & 对bool特征\emph{B...}进行逻辑\emph{与}运算 \\
        \texttt{disjunction<B...>} & 对bool特征\emph{B...}进行逻辑\emph{或}运算 \\
        \texttt{negation<B>}       & 对bool特征B进行\emph{非}运算             \\
        \hline
    \end{tabular}
    \caption{组合其他类型特征的类型特征}
    \label{t21.2}
\end{table}

它们的一大应用就是通过组合现有类型特征来定义新的类型特征。
例如,你可以轻松的定义一个检查某个类型是否是“指针”(原生指针,成员函数指针,或者空指针)的特征:
\begin{lstlisting}
    template<typename T>
    struct IsPtr : std::disjunction<std::is_null_pointer<T>,
                                    std::is_member_pointer<T>,
                                    std::is_pointer<T>> {
    };
\end{lstlisting}

现在我们在一个\nameref{ch10}中使用这个新的特征:
\begin{lstlisting}
    template<typename T>
    void foo(T x)
    {
        if constexpr(IsPtr<T>) {
            ... // 处理是指针的情况
        }
        else {
            ... // 处理不是指针的情况
        }
    }
\end{lstlisting}
作为另一个例子,我们可以定义一个检查指定类型是否是整数或者枚举但又不是bool的类型特征:
\begin{lstlisting}
    template<typename T>
    struct IsIntegralOrEnum : std::conjunction<std::disjunction<std::is_integral<T>,
                                                                std::is_enum<T>>,
                                               std::negation<std::is_same<T, bool>>> {
    };
\end{lstlisting}
这里,类似于计算
\begin{lstlisting}
    (is_integral<T> || is_enum<T>) && !is_same<T, bool>
\end{lstlisting}
这么写的一个好处是\texttt{std::conjunction<>}和\texttt{std::disjunction<>}
会\emph{短路求值}bool表达式,这意味着当\emph{conjunction}出现第一个\texttt{false}或者
\emph{disjunction}出现第一个\texttt{true}时就会停止计算。
这节省了编译时间,甚至因为短路求值可以在某些情况下使原本无效的代码变得有效。
\footnote{感谢Howard Hinnant指出这一点。}
例如,如果像下面这样使用不完全类型:
\begin{lstlisting}
    struct X {
        X(int);     // 从int转换而来
    };

    struct Y;       // 不完全类型
\end{lstlisting}
下面的静态断言会失败,因为对于不完全类型\texttt{is\_constructible}会陷入未定义行为
(尽管有些编译器接受这样的代码):
\begin{lstlisting}
    // 未定义行为
    static_assert(std::is_constructible<X, int>{} || std::is_constructible<Y, int>{},
                  "can't init X or Y from int");
\end{lstlisting}
下面使用\texttt{std::disjunction}的替代版保证不会失败,
因为当\texttt{is\_constructible<X, int>}返回\texttt{true}后求值就会停止:
\begin{lstlisting}
    // OK:
    static_assert(std::disjunction<std::is_constructible<X, int>,
                                   std::is_constructible<Y, int>>{},
                  "can't init X or Y from int");
\end{lstlisting}


\section{后记}
标准类型特征的变量模板由Stephan T. Lavavej于2014年在\url{https://wg21.link/n3854}
中首次提出。它们最终因Alisdair Meredith发表于\url{https://wg21.link/p0006r0}
的提案而被Library Fundamentals TS采纳。

类型特征\texttt{std::is\_aggregate<>}作为美国国家机构对C++17标准的注释
引入(见\url{https://wg21.link/lwg2911})。

\texttt{is\_swappable}特征家族由Andrew Morrow在\url{https://wg21.link/n3619}
中首次提出。最终被接受的是Daniel Krügler发表于\url{https://wg21.link/p0185r1}的提案。

\texttt{std::has\_unique\_object\_representations<>}由Michael Spencer在
\url{https://wg21.link/p0258r0}中以名称\texttt{is\_contiguous\_layout}首次提出,
最终被接受的是Michael Spencer在\url{https://wg21.link/p0258r2}中的提案。

\texttt{is\_invocable}特征家族由Agustin Berge在\url{https://wg21.link/n4446}中首次提出。\\
它和\texttt{std::invoke\_result\_t<>}最终因Daniel Krügler、Pablo Halpern、
Jonathan Wakely发表于\url{https://wg21.link/p0604r0}的提案而一起被标准库接受。

bool类型特征的逻辑操作由Jonathan Wakely在\url{https://wg21.link/p0013r0}中首次提出。
最终被接受的是Jonathan Wakely发表于\url{https://wg21.link/p0013r1}的提案。