\section{\texttt{std::byte}}\label{ch18}
通过\texttt{std::byte},C++17引入了一个类型来代表内存的最小单位:字节。
\texttt{std::byte}本质上代表一个字节的值,但并不能进行数字或字符的操作,
也不对每一位进行解释。对于不需要数字计算和字符序列的场景,这样会更加类型安全。

然而你,注意\texttt{std::byte}实现的大小和\texttt{unsigned char}一样,这意味着
它并不保证是8位,可能会更多。

\subsection{使用\texttt{std::byte}}
下面的代码展示了\texttt{std::byte}的核心能力:
\begin{lstlisting}
    #include <cstddef>  // for std::byte

    std::byte b1{0x3F};
    std::byte b2{0b1111'0000};

    std::byte b3[4] {b1, b2, std::byte{1}}; // 4个字节(最后一个是0)

    if (b1 == b3[0]) {
        b1 <<= 1;
    }

    std::cout << std::to_integer<int>(b1) << '\n';  // 输出:126
\end{lstlisting}
这里,我们定义了两个初始值不同的字节。\texttt{b2}的初始化使用了两个C++14引入的特性:
\begin{itemize}[leftmargin=*]
    \item 前缀\texttt{0b}允许定义二进制字面量
    \item \emph{数字分隔符}\texttt{'}可以增强数字字面量的可读性
    (它可以被放置在数字字面量中任意两个数字之间)。
\end{itemize}
注意列表初始化(使用花括号初始化)是唯一可以直接初始化\texttt{std::byte}对象的方法。
所有其他的形式都不能编译:
\begin{lstlisting}
    std::byte b1{42};       // OK(因为自从C++17起所有枚举都有固定的底层类型)
    std::byte b2(42);       // ERROR
    std::byte b3 = 42;      // ERROR
    std::byte b4 = {42};    // ERROR
\end{lstlisting}
这是将\texttt{std::byte}实现为枚举类型的一个直接后果。
花括号初始化使用了新的\hyperref[ch8.3]{用整数值初始化有作用域的枚举}特性。

这里不能使用隐式类型转换,这意味着你必须显式对整数值进行转换才能初始化字节数组:
\begin{lstlisting}
    std::byte b5[] {1};             // ERROR
    std::byte b6[] {std::byte{1}};  // OK
\end{lstlisting}
如果没有初始化,\texttt{std::byte}将是栈上的值未定义的对象:
\begin{lstlisting}
    std::byte b;    // 值未定义
\end{lstlisting}
像通常一样(除了原子类型),你可以使用花括号强制初始化为每一位为0:
\begin{lstlisting}
    std::byte b{};  // 等价于b{0}
\end{lstlisting}
\texttt{std::to\_integer<>}允许你将\texttt{std::byte}对象转换为整数值(包括\texttt{bool}
和\texttt{char}类型)。如果没有转换,将不能使用输出运算符。注意因为这个转换函数是模板,
所以你需要使用带有\texttt{std::}的完整名称:
\begin{lstlisting}
    std::cout << b1;    // ERROR
    std::cout << to_integer<int>(b1);       // ERROR(ADL在这里不起作用)
    std::cout << std::to_integer<int>(b1);  // OK
\end{lstlisting}
也可以使用using声明(但请只在局部作用域中这么做):
\begin{lstlisting}
    using std::to_integer;
    ...
    std::cout << to_integer<int>(b1);       // OK
\end{lstlisting}
如果要将\texttt{std::byte}用作布尔值也需要这样的转换。例如:
\begin{lstlisting}
    if (b2) ...                         // ERROR
    if (b2 != std::byte{0}) ...         // OK
    if (to_integer<bool>(b2)) ...       // ERROR(ADL在这里不起作用)
    if (std::to_integer<bool>(b2)) ...  // OK
\end{lstlisting}
因为\texttt{std::byte}被实现为底层类型是\texttt{unsigned char}的枚举类型,
所以它的大小总是1:
\begin{lstlisting}
    std::cout << sizeof(b);             // 总是1
\end{lstlisting}
它的位数依赖于底层类型\texttt{unsigned char}的位数,
你可以通过标准数字限制来获取位数:
\begin{lstlisting}
    std::cout << std::numeric_limits<unsigned char>::digits; // std::byte的位数
\end{lstlisting}
这等价于:
\begin{lstlisting}
    std::cout <<
        std::numeric_limits<std::underlying_type_t<std::byte>>::digits;
\end{lstlisting}
大多数时候结果是8,但在有些平台上可能不是。

\subsection{\texttt{std::byte}类型和操作}
这一节详细描述\texttt{std::byte}类型和操作。

\subsubsection{\texttt{std::byte}类型}
在头文件\texttt{<cstddef>}中,C++标准库以如下方式定义了\texttt{std::byte}:
\begin{lstlisting}
    namespace std {
        enum class byte : unsigned char {
        };
    }
\end{lstlisting}
也就是说,\texttt{std::byte}不是别的,只是一个带有一些位运算符操作的有作用域的枚举类型:
\begin{lstlisting}
    namespace std {
        ...
        template<typename IntType>
        constexpr byte  operator<<  (byte  b, IntType shift) noexcept;
        template<typename IntType>
        constexpr byte& operator<<= (byte& b, IntType shift) noexcept;
        template<typename IntType>
        constexpr byte  operator>>  (byte  b, IntType shift) noexcept;
        template<typename IntType>
        constexpr byte& operator>>= (byte& b, IntType shift) noexcept;

        constexpr byte& operator|= (byte& l, byte r) noexcept;
        constexpr byte  operator|  (byte  l, byte r) noexcept;
        constexpr byte& operator&= (byte& l, byte r) noexcept;
        constexpr byte  operator&  (byte  l, byte r) noexcept;
        constexpr byte& operator^= (byte& l, byte r) noexcept;
        constexpr byte  operator^  (byte  l, byte r) noexcept;
        constexpr byte  operator~  (byte b) noexcept;

        template<typename IntType>
        constexpr IntType to_integer (byte b) noexcept;
    }
\end{lstlisting}

\subsubsection{\texttt{std::byte}操作}
表\hyperref[t18.1]{\texttt{std::byte}的操作}列出了\texttt{std::byte}的所有操作。
\begin{table}[ht]
    \begin{tabular}{l|p{0.5\textwidth}}
        \hline
        \textbf{操作} & \textbf{效果} \\
        \hline
        \emph{构造函数} & 创建一个字节对象(调用默认构造函数时值未定义) \\
        \emph{析构函数} & 销毁一个字节对象(什么也不做) \\
        \texttt{=} & 赋予新值 \\
        \texttt{==、!=、<、<=、>、>=} & 比较字节对象 \\
        \texttt{<<、>>、|、\&、\textasciicircum、\textasciitilde} & 二元位运算符    \\
        \texttt{<<=、>>=、|=、 \&=、\textasciicircum =} & 修改自身的位运算符 \\
        \texttt{to\_integer<T>()} & 把字节对象转换为整数类型\texttt{T} \\
        \texttt{sizeof()} & 返回1 \\
        \hline
    \end{tabular}
    \caption{\texttt{std::byte}的操作}
    \label{t18.1}
\end{table}

\subsubsection*{转换为整数类型}
用\texttt{to\_integer<>()}可以把\texttt{std::byte}转换为任意基本整数类型
(\texttt{bool}、字符类型或者整数类型)。这也是必须的,
例如为了将\texttt{std::byte}和整数值比较或者将它用作条件:
\begin{lstlisting}
    if (b2) ...                         // ERROR
    if (b2 != std::byte{0}) ...         // OK
    if (to_integer<bool>(b2)) ...       // ERROR(ADL在这里不生效)
    if (std::to_integer<bool>(b2)) ...  // OK
\end{lstlisting}
另一个例子使用它的例子是\hyperref[ch18.2.2.2]{\texttt{std::byte} I/O}。

\texttt{to\_integer<>()}使用\texttt{static\_cast}来把\texttt{unsigned char}
转换为目标类型。例如:
\begin{lstlisting}
    std::byte ff{0xFF};
    std::cout << std::to_integer<unsigned int>(ff); // 255
    std::cout << std::to_integer<int>(ff);          // 也是255
    std::cout << static_cast<int>(std::to_integer<signed char>(ff)); // -1
\end{lstlisting}

\subsubsection*{\texttt{std::byte}的I/O}\label{ch18.2.2.2}
\texttt{std::byte}没有定义输入和输出运算符,因此不得不把它转换为整数类型再进行I/O:
\begin{lstlisting}
    std::byte b;
    ...
    std::cout << std::to_integer<int>(b);   // 以十进制值打印出值
    std::cout << std::hex << std::to_integer<int>(b); // 以十六进制打印出值
\end{lstlisting}
通过使用\texttt{std::bitset<>},你可以以二进制输出值(一串位序列):
\begin{lstlisting}
    #include <cstddef>  // for std::byte
    #include <bitset>   // for std::bitset
    #include <limits>   // for std::numeric_limits

    std::byte b1{42};
    using ByteBitset =
        std::bitset<std::numeric_limits<unsigned char>::digits>;
    std::cout << ByteBitset{std::to_integer<unsigned>(b1)};
\end{lstlisting}
上例中using声明定义了一个位数和\texttt{std::byte}相同的bitset类型,
之后把字节对象转换为整数来初始化一个这种类型的对象,最后输出了该对象。
最后值42将会有如下输出(假设一个\texttt{char}是8位):
\begin{lstlisting}
    00101010
\end{lstlisting}
另外,你可以使用\texttt{std::underlying\_type\_t<std::byte>}代替\texttt{unsigned char},
这样using声明的目的将更明显。

你也可以使用这种方法把\texttt{std::byte}的二进制表示写入一个字符串:
\begin{lstlisting}
    std::string s = ByteBitset{std::to_integer<unsigned>(b1)}.to_string();
\end{lstlisting}
如果你已经有了一个字符序列,你也可以像下面这样使用\nameref{ch31.2.2}:
\footnote{感谢Daniel Krügler指出这一点。}
\begin{lstlisting}
    #include <charconv>
    #include <cstddef>

    std::byte b1{42};
    // 译者注:此处原文写的是
    // int value = 42;
    // 应是作者笔误

    char str[100];
    std::to_chars_result res =
        std::to_chars(str, str+99, std::to_integer<int>(b1), 2);
    *res.ptr = '\0';    // 确保最后有一个空字符结尾
\end{lstlisting}
注意这种形式将不会写入前导0,这意味着对于值42,最后的结果是(假设一个\texttt{char}有8位):
\begin{lstlisting}
    101010
    // 译者注:此处原文写的是
    // 1111110
    // 应是作者笔误
\end{lstlisting}
可以使用相似的方式进行输入:以整数、字符串或bitset类型读入并进行转换。
例如,你可以像下面这样实现读入字节对象的二进制表示的输入运算符:
\begin{lstlisting}
    std::istream& operator>> (std::istream& strm, std::byte& b)
    {
        // 读入一个bitset
        std::bitset<std::numeric_limits<unsigned char>::digits> bs;
        strm >> bs;
        // 如果没有失败就转换为std::byte
        if (!std::cin.fail()) {
            b = static_cast<std::byte>(bs.to_ulong());  // OK
        }
        return strm;
    }
\end{lstlisting}
注意我们必须使用\texttt{static\_cast<>()}来把bitset转换成的unsigned long转换为
\texttt{std::byte},列表初始化将不能工作,因为会发生窄化:
\footnote{使用gcc/g++时,如果没有编译选项\texttt{-pedantic-errors},
窄化初始化也可能通过编译。}
\begin{lstlisting}
    b = std::byte{bs.to_ulong()};   // ERROR:发生窄化
\end{lstlisting}
并且我们也没有其他的初始化方法了。

另外,你也可以使用\nameref{ch31.2.1}来从给定的字符序列读取:
\begin{lstlisting}
    #include <charconv>

    const char* str = "101001";
    int value;
    std::from_chars_result res =
        std::from_chars(str, str+6, // 要读取的字符范围
                        value,      // 读取后存入的对象
                        2);         // 2进制
\end{lstlisting}

\subsection{后记}
Neil MacIntosh在发表于\url{https://wg21.link/p0298r0}的提案中首次提出\texttt{std::byte}。
最终被接受的提案由Neil MacIntosh发表于\url{https://wg21.link/p0298r3}。

